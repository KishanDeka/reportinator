\documentclass{../_layouts/ieeeconf}

\begin{document}

\title{Test}

\author{Spandan Anupam}

\date{\today}
\maketitle

\begin{abstract}
abc blah blah

\end{abstract}
\section{Objective:}
The objectives of this two part disjoint experiment series will be to: -
Study Electron Spin Resonance spectra for a given sample, and explain
the number, and position of peaks - Perform experiments with the ExpEyes
system, one being studying the induced voltage when a small magnet is
dropped through a coil, and the other being looking at how the voltage
pulses when a led at a particular frequency is shown to a photodiode

\section{Theory:}
\hypertarget{electron-spin-resonance}{%
\subsection{Electron Spin Resonance:}\label{electron-spin-resonance}}

Electron spin resonance (ESR) or Electron paramagentic spin resonance
(EPSR) is a spectroscopy method to study materials with unpaired
electrons. The basic concept here, being that we see a particular energy
being assigned to electrons, when kept in a magnetic field. These being
spin half paricles, we will either have the electron aligning parallel
(\(m_s = 1/2\)) or antiparallel (\(m_s = -1/2\)) to the field. The
energy assigned is given by: \begin{equation}
    E = m_s g_e \mu_B B_0
\end{equation} where: \newline E refers to the energy \newline \(m_s\)
refers to the magneitc component of the spin \newline \(g_e\) refers to
the lande g factor \newline \(\mu_B\) refers to a Bohr magneton
\newline and \(B_0\) is the applied magnetic field Now an electron can
of course move between these two states by absorbing or emmiting a
photon, with energy \(h\nu\). So we get another equation from here:
\begin{equation}
    h\nu=m_s g_e \mu_B B_0
\end{equation} where: \newline \(\nu\) is the wavenumber of the exciting
RF wave. For our case, we are keeping the frequency of the RF wave
constant, and changing the magnetic field. We will, at some point, reach
an energy where the energy is absorbed the most, due to the transiton.
We are assuming here that most of the electrons are in the lower energy
level, in a normal case. We here, are attenuating a DC voltage through
the coil with a small 50 Hz AC voltage, so that the magnetic field
sweeps from \(I_{DC}\)-\(I_{AC max}\) to \(I_{DC}\)+\(I_{AC max}\). This
will contain the absorbance energy. \begin{equation}
H_0 = \frac{2\sqrt{2}H}{P}Q
\end{equation} \begin{equation}\label{key}
H=\frac{32\pi n}{10\sqrt{125}a}I
\end{equation} \begin{equation}\label{key}
H_0=2\sqrt{2}\frac{32\pi n}{10\sqrt{125}a}\frac{QI}{P}
\end{equation} Substituting the value of a = 7.6 cm, n = 500 turns we
get \begin{equation}\label{key}
Q=\frac{10\sqrt{125}a PH_0}{64\sqrt{2}\pi n }\frac{1}{I}=\frac{PH_0}{168}\frac{1}{I} 
\end{equation} From the plot of Q Vs 1/I , the slope gives :
\begin{equation}\label{key}
    \frac{PH_0}{168}=slope \implies H_0=slope \times \frac{168}{P}
\end{equation} \begin{equation}\label{eq}
g=\frac{h \nu}{H_0 \mu_0} = 4.25\times10^{-9} \frac{P \nu}{slope}
\end{equation} \#\# ExpEyes: This is basically an interface which
changes the analog signals we get, into digital. The ``digital
oscilloscope'' gives us the freedom to do some simple physics
experiments with a greater ease. The experiment mostly consisted of
familatrisation of oneself with the instruments. \#\# Electromagnetic
Induction: Electromagentic Induction is the effect where we see a
current being induced in a changin magnetic field. From (Najiya Maryam,
2014), we see that the expression for the induced current for a magnet
falling through a coil is given by: \begin{align}
    EMF = &\frac{2\mu_o m}{2\pi}(-z_o+0.5\times gt^2) \\ \nonumber
    &\times (R^2+(-z_o+0.5\times gt^2)^2)^\frac{-5}{2}\times gt
\end{align} where: \newline EMF is the induced voltage \newline \(m\)
refers to the magentic moment of the small magnet \newline g is the
acceleration due to gravity \newline \(\mu_o\) refers to a permittivity
of free space \newline t is the time \newline R is the radius of the
coil \newline N is the number of turns \newline \(z_o\) is the height
from which the magnet is dropped Our job here, will be to calculate the
magnetic moment of the small magnet by fitting the experimental data as
close to the theoretical data. We will of course, have som deviations
considering that the magnet does not stay straight at all times, there
is air resistance, and many other factors that we considered. I took the
liberty of matching the ``0s'\,' of the graphs and callibrating the
digital data by hand, and not including it in the data listed.

\subsubsection{Induced Voltage in a photodiode:}

This is relatively simple, we just need to observe what the voltage from
the input to the LED is, and how that is affecting the photodiode. The
plot is given in the observations section.

\section{Observations:}
\begin{table}[H]
    \centering
    \begin{tabular}{@{}cccc@{}}
        \toprule
        Temperature (C) & Temperature (K) & Capacitance (nF) & \epsilon \\
        \midrule
        23.5 & 296.5 & 635 & 4.04E-09 \\
        51 & 324 & 644 & 4.10E-09 \\
        56 & 329 & 650 & 4.14E-09 \\
        60 & 333 & 657 & 4.18E-09 \\
        65 & 338 & 658 & 4.19E-09 \\
        70 & 343 & 673 & 4.29E-09 \\
        75 & 348 & 695 & 4.43E-09 \\
        80 & 353 & 726 & 4.62E-09 \\
        85 & 358 & 767 & 4.89E-09 \\
        90 & 363 & 820 & 5.22E-09 \\
        95 & 368 & 905 & 5.76E-09 \\
        100 & 373 & 1020 & 6.50E-09 \\
        105 & 378 & 1120 & 7.13E-09 \\
        110 & 383 & 1100 & 7.01E-09 \\
        112 & 385 & 1050 & 6.69E-09 \\
        113 & 386 & 1028 & 6.55E-09 \\
        114 & 387 & 1000 & 6.37E-09 \\
        115 & 388 & 980 & 6.24E-09 \\
        116 & 389 & 958 & 6.10E-09 \\
        117 & 390 & 933 & 5.94E-09 \\
        118 & 391 & 910 & 5.80E-09 \\
        119 & 392 & 889 & 5.66E-09 \\
        120 & 393 & 870 & 5.54E-09 \\
        graph(1,2) &  &  &  \\
        \bottomrule
    \end{tabular}
\end{table}
\begin{table}[H]
    \centering
    \begin{tabular}{@{}cc@{}}
        \toprule
        test1 & test2 \\
        \midrule
        0 & 0 \\
        1 & 2 \\
        2 & 4 \\
        3 & 6 \\
        4 & 8 \\
        5 & 10 \\
        6 & 12 \\
        7 & 14 \\
        8 & 16 \\
        9 & 18 \\
        \bottomrule
    \end{tabular}
\end{table}
\section{Graphs:}
\section{Conclusion:}
We see that experiments turned out to be as expected, I estimate the
magnetic moment to be 1.6 A.m\(^2\). Substituting: P = 5,
\(\nu = 14.37 \times 10^6 Hz\) and slope = 0.195 in eq \ref{eq} we also
get \(g = 1.565\), which is pretty close to it's actual value.

\section{Precautions:}
\begin{itemize}
\tightlist
\item
  Care must be taken and the knobs adjusted to keep the phase zero at
  each change in current for ESR
\item
  The magnet must be dropped as vertically as possible
\end{itemize}

\section{References:}
Najiya Maryam, K.M. (2014). EM induction experiment to determine the
moment of a magnet. Physics Education, 49(3), pp.319--325.

\end{document}